\documentclass[final]{beamer}

\usepackage[scale=1.35]{beamerposter} % Use the beamerposter package for laying out the poster

\usetheme{confposter} % Use the confposter theme supplied with this template

\setbeamercolor{block title}{fg=ngreen,bg=white} % Colors of the block titles
\setbeamercolor{block body}{fg=black,bg=white} % Colors of the body of blocks
\setbeamercolor{block alerted title}{fg=white,bg=dblue!70} % Colors of the highlighted block titles
\setbeamercolor{block alerted body}{fg=black,bg=dblue!10} % Colors of the body of highlighted blocks
% Many more colors are available for use in beamerthemeconfposter.sty
%-----------------------------------------------------------
% Define the column widths and overall poster size
% To set effective sepwid, onecolwid and twocolwid values, first choose how many columns you want and how much separation you want between columns
% In this template, the separation width chosen is 0.024 of the paper width and a 4-column layout
% onecolwid should therefore be (1-(# of columns+1)*sepwid)/# of columns e.g. (1-(4+1)*0.024)/4 = 0.22
% Set twocolwid to be (2*onecolwid)+sepwid = 0.464
% Set threecolwid to be (3*onecolwid)+2*sepwid = 0.708

\newlength{\sepwid}
\newlength{\onecolwid}
\newlength{\twocolwid}
\newlength{\threecolwid}
\setlength{\paperwidth}{60in} % A0 width: 46.8in
\setlength{\paperheight}{42in} % A0 height: 33.1in
\setlength{\sepwid}{0.024\paperwidth} % Separation width (white space) between columns
\setlength{\onecolwid}{0.22\paperwidth} % Width of one column
\setlength{\twocolwid}{0.464\paperwidth} % Width of two columns
\setlength{\threecolwid}{0.708\paperwidth} % Width of three columns
\setlength{\topmargin}{-0.5in} % Reduce the top margin size
%-----------------------------------------------------------

\usepackage{graphicx}  % Required for including images
\graphicspath{{./graphics/}}
\usepackage{booktabs} % Top and bottom rules for tables

%----------------------------------------------------------------------------------------
%	TITLE SECTION 
%----------------------------------------------------------------------------------------

\title{\textrm{Development of \textbf{FWI4GPR}, an open-source package for Full-Waveform Inversion of common-offset GPR data}} % Poster title

\author{\textrm{\textit{Sajad Jazayeri}, Sarah Kruse}} % Author(s)
\institute{\textrm{~~School of Geosciences, University of South Florida}} % Institution(s)
%----------------------------------------------------------------------------------------

\setbeamertemplate{headline}{
	\leavevmode
	\begin{columns}
		\begin{column}{0.85\linewidth}
			\vskip1cm

			\usebeamercolor{title in headline}{\color{jblue}\Huge{\textbf{\inserttitle}}\\[0.5ex]}
%			\hspace{2cm}
			\usebeamercolor{author in headline}{\color{fg}\Large{\insertauthor}}
			\hspace{27cm}
			{\color{jblue}\large{\textrm{contact:~~~\textbf{\href{mailto:sjazayeri@mail.usf.edu}{sjazayeri@mail.usf.edu}}}}}\hspace{21cm}\usebeamercolor{institute in headline}{\color{fg}\large{\hfill\insertinstitute}}%\\[1ex]}
			\hspace{16cm}
			{\color{jblue}\large{\textrm{webpage:~\textbf{ \href{http://sjazayeri.myweb.usf.edu}{http://sjazayeri.myweb.usf.edu}}}}\\[0.5ex]}
			\vskip1cm
		\end{column}
		\begin{column}{0.15\linewidth}
			~~~~~~~~\includegraphics[width=1.3\linewidth]{USF.png}
		\end{column}
	\end{columns}
	\vspace{1cm}
	\hspace{3cm}\begin{beamercolorbox}[wd=145cm,colsep=0.15cm]{cboxb}\end{beamercolorbox} % changed inches to 3cm less than max page width
	\vspace{0.5cm}
}


\begin{document}

\addtobeamertemplate{block end}{}{\vspace*{0.5ex}} % White space under blocks
\addtobeamertemplate{block alerted end}{}{\vspace*{0.5ex}} % White space under highlighted (alert) blocks

\setlength{\belowcaptionskip}{1ex} % White space under figures
\setlength\belowdisplayshortskip{2ex} % White space under equations

\begin{frame}[t] % The whole poster is enclosed in one beamer frame

\begin{columns}[t] % The whole poster consists of three major columns, the second of which is split into two columns twice - the [t] option aligns each column's content to the top

\begin{column}{\sepwid}\end{column} % Empty spacer column

\begin{column}{\onecolwid} % The first column

%----------------------------------------------------------------------------------------
%	INTRODUCTION
%----------------------------------------------------------------------------------------

\begin{block}{Introduction}

We introduce a package for full-waveform inversion (FWI) of Ground Penetrating Radar (GPR) data based on a
combination of open-source programs. FWI is non-linear data-fitting procedure that aims at obtaining detailed estimates of subsurface properties from data using an iterative process. It requires a good starting model, based on direct knowledge of
field conditions or on traditional ray-based inversion methods and an optimal source wavelet. With a good starting model and wavelet, the FWI can improve resolution of selected subsurface features. The package will be made available for general use in educational and research activities.

\end{block}

%----------------------------------------------------------------------------------------
%	OBJECTIVES
%----------------------------------------------------------------------------------------
\vspace*{2cm}

\begin{alertblock}{Components}
	
The package has five main components, FWI is performed in an  iterative process and needs initial models preparation. Here are the components:
	\begin{enumerate}
		\item \hspace{1cm} \texttt{Forward Modeler}
		\item \hspace{1cm} \texttt{Inversion Algorithm}
		\item \hspace{1cm} \texttt{Ray-based analyzer}
		\item \hspace{1cm} \texttt{Source Wavelet (SW) estimator}
		\item \hspace{1cm} \texttt{3D to 2D converter} 
	\end{enumerate}
	
\end{alertblock}

\vspace*{2cm}

%------------------------------------------------
\begin{block}{\textsc{\texttt{Forward Modeler}}}
	
\begin{figure}
	\includegraphics[width=0.7\linewidth]{gprmax.png}
	\vspace{1cm}
	\includegraphics[width=1\linewidth]{gprmax_image.png}
	\caption{gprMax sample input and ASCII output \cite{gprmax}}
\end{figure}
	
\end{block}

%----------------------------------------------------------------------------------------

\end{column} % End of the first column

\begin{column}{\sepwid}\end{column} % Empty spacer column

\begin{column}{\twocolwid} % Begin a column which is two columns wide (column 2)

\begin{columns}[t,totalwidth=\twocolwid] % Split up the two columns wide column

\begin{column}{\onecolwid}\vspace{-.6in} % The first column within column 2 (column 2.1)

%----------------------------------------------------------------------------------------
%	MATERIALS
%----------------------------------------------------------------------------------------

\begin{block}{\textsc{\texttt{Inversion Algorithm}}}

\begin{figure}
	\includegraphics[width=0.8\linewidth]{PEST_flat.jpg}
	\vspace{1cm}
	\includegraphics[width=1.05\linewidth]{pest_image.png}
	\caption{PEST sample control and template files\cite{pest}}
\end{figure}

\end{block}

%----------------------------------------------------------------------------------------

\end{column} % End of column 2.1

\begin{column}{\onecolwid}\vspace{-.6in} % The second column within column 2 (column 2.2)

%----------------------------------------------------------------------------------------
%	METHODS
%----------------------------------------------------------------------------------------

\begin{block}{\textsc{\texttt{ SW estimator}}}
	
Apply \textbf{Sparse Blind Deconvolution} to estimate  waveform. Data is convolution product of the source wavelet and the reflecitivity series (impulse response).

%\vspace*{1.5cm}
	\begin{figure}
		\includegraphics[width=1\linewidth]{synthtic_model_SBD.png}
		\vspace{-3mm}
		\line(1,0){900}
		\vspace{1cm}
		\includegraphics[width=0.98\linewidth]{SBD_synthetic_refl.png}
		\caption{Sparsity Based Source wavelet estimation, synthetic case \cite{jazayeri2017sparse}}
	\end{figure}
	
\end{block}

%----------------------------------------------------------------------------------------

\end{column} % End of column 2.2

\end{columns} % End of the split of column 2 - any content after this will now take up 2 columns width

%----------------------------------------------------------------------------------------
%	IMPORTANT RESULT
%----------------------------------------------------------------------------------------

\begin{alertblock}{Key factors}

FWI is an iterative process, where the \textbf{starting models} are very important to increase the success of the process. Also, no FWI process would be possibly successful without a good \textbf{source wavelet}.

\end{alertblock} 

%----------------------------------------------------------------------------------------

\begin{columns}[t,totalwidth=\twocolwid] % Split up the two columns wide column again

\begin{column}{\onecolwid} % The first column within column 2 (column 2.1)

%----------------------------------------------------------------------------------------
%	MATHEMATICAL SECTION
%----------------------------------------------------------------------------------------

\begin{block}{\textsc{\texttt{Ray-based analyzer}}}

Use ray-based analysis to estimate the initial parameters. We use travel times of the peak amplitudes of the diffraction hyperbola within a \textbf{least-squares} approach to estimate model parameters.  

\begin{figure}
	\includegraphics[width=0.7\linewidth]{ray-based.png}
	\caption{Ray-based hyperbola fitting on GPR data over a pipe \cite{jazayeri2017}}
\end{figure}

\end{block}

%----------------------------------------------------------------------------------------

\end{column} % End of column 2.1

\begin{column}{\onecolwid} % The second column within column 2 (column 2.2)

%----------------------------------------------------------------------------------------
%	RESULTS
%----------------------------------------------------------------------------------------
\vspace{-2cm}
\begin{block}{}%{\textsc{\texttt{Ray-based analyzer}}}
\begin{figure}
	\includegraphics[width=0.4\linewidth]{SBD_pipe_sw.png}
	
	\includegraphics[width=0.7\linewidth]{SBD_pipe_refl.png}
	\caption{\textrm{SBD resulted source wavelet and reflectivity series~\cite{jazayeri2017sparse}}}
\end{figure}
\end{block}


%----------------------------------------------------------------------------------------

\end{column} % End of column 2.2

\end{columns} % End of the split of column 2

\end{column} % End of the second column

\begin{column}{\sepwid}\end{column} % Empty spacer column

\begin{column}{\onecolwid} % The third column

\begin{block}{\textsc{\texttt{3D to 2D Converter}}}
	
	Simulate 2D line-source generated waveforms that would be equivalent to those observed in the 3D data, by convolving data in the time domain with $\sqrt{t}$ where $t$ is travel time.
	
\end{block}

\vspace{3cm}

\begin{block}{Example}
	
	\begin{figure}
		\includegraphics[width=1\linewidth]{FWI_pipes_frames.png}
		\caption{Collected and synthetic data fit after ray-based, SW correction and FWI \cite{jazayeri2017}}
	\end{figure}
	
\end{block}

%----------------------------------------------------------------------------------------
%	ACKNOWLEDGEMENTS
%----------------------------------------------------------------------------------------
\vspace{1.5cm}

\setbeamercolor{block title}{fg=ngreen,bg=white} % Change the block title color

\begin{block}{Acknowledgements}
	
	\small{\rmfamily{We would like to thank Mr. A. Ebrahimi for his useful comments on SBD, Drs. J. Doherty, A. Giannopoulos and C. Warren for making their codes open-source.}} \\
	
\end{block}

\vspace{1.5cm}

\begin{block}{References}

\nocite{*} % Insert publications even if they are not cited in the poster
\small{\bibliographystyle{unsrt}
\bibliography{references}\vspace{0.25in}}

\end{block}



%----------------------------------------------------------------------------------------
%	CONTACT INFORMATION
%----------------------------------------------------------------------------------------

\setbeamercolor{block alerted title}{fg=white,bg=jblue!70} % Change the alert block title colors
\setbeamercolor{block alerted body}{fg=black,bg=jblue!5} % Change the alert block body colors


%----------------------------------------------------------------------------------------

\end{column} % End of the third column

\end{columns} % End of all the columns in the poster

\end{frame} % End of the enclosing frame

\end{document}
